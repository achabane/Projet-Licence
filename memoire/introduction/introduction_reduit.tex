\introduction
      \paragraph{}
      \small{
	}
    
    \paragraph{Problématique\\ \\}
    \small{ 
	 Le nombre d’utilisateurs de smartphone est chaque jour grandissant et la plupart de
ces outils de communication sont aujourd’hui connectés. Cependant, on enregistre souvent
des plaintes des utilisateurs en particulier ceux concernant la consommation de données
mobiles de leurs téléphones. La plus par accusent les opérateurs GSM, d’autres se plaignent de ne pouvoir contrôler certaines mises à jour de certaines applications, et d'autres même du fonctionnement arrière-plan de certaines applications etc. Il serait intéressant de disposer d’un gestionnaire de la consommation d’internet de chaque
application de nos téléphones afin que les utilisateurs de smartphones puissent facilement avoir le contrôle sur la consommation des données et de planifier la consommation de données sur une période données.
  
	}

    \paragraph{Objectif\\ \\}
    \small{ 
	  Notre projet à pour objectif principal de mettre à la disposition des utilisateurs de smartphones, une application android
	  leurs permettant de controler leurs consommation en terme de forfait internet.
      
	  Plus précisement il s'agira de permettre aux utilisateurs de pouvoir:
	
    \begin{itemize}
	\item évaluer la consommation internet (données mobiles et wifi) pour chaque application,
	\item définir des seuils de consommation de chaque application,
	\item bloquer et autoriser l'accès à internet (données mobiles et/ou wifi) de certains applications,
	\item planifier la consommation de chaque application (ou du téléphone) sur une période donnée,
        \item spécifier à certaines applications de n’utiliser que du wifi,
        \item donner des statistiques de consommation.
    \end{itemize}
    }

    \paragraph{\\}
	\small{
	  Le présent mémoire fait le point de nos travaux et comporte trois (3) chapitres.
	
	  La premier présente une revue de littérature sur la géolocalisation mobile
      sur la base de laquelle nous avons élaboré notre problématique.
    
	  Dans le deuxième chapitre, nous présentons les choix techniques opérés en
      vue de la conception et de la réalisation de la solution proposée.
    
	  Le troisième chapitre fait une analyse critique des résultats issus de nos simulations 
      après les avoir exposés.
	}


